\section{Assessment}
The course is tested with two exams:
A series of \glspl{assignment} which have to be handed in, but won't be graded. There will be an \gls{oral}, which is based on the \glspl{assignment}
and a written exam. The final grade is determined as follows: \\

\texttt{if \gls{exam}-grade $ >= 75\% $ then return \gls{oral}-grade else return 0}

\paragraph*{Motivation for grade}
A professional software developer is required to be able to program code which is, at the very least, \textit{correct}.

In order to produce correct code, we expect students to show:
\begin{inparaenum}[\itshape i\upshape)]
\item a foundation of knowledge about how a programming language actually works in connection with a simplified concrete model of a computer;
\item fluency when actually writing the code.
\end{inparaenum}

The quality of the programmer is ultimately determined by his actual code-writing skills, therefore the written exam will require you to write code; this ensures that each student is able to show that his work is his own and that he has adequate understanding of its mechanisms.


\subsection{Theoretical examination \modulecode}
The general shape of a \gls{exam} for \texttt{\modulecode} is made up of a short series of highly structured open questions.
In each exam the content of the questions will change, but the structure of the questions will remain the same.
For the structure (and an example) of the theoretical exam, see the appendix.


\subsection{Practical examination \modulecode}
There is an assignment for each pattern covered in the lessons that is mandatory. The assignment asks you to implement a GUI system in the fashion of Windows Form with an immediate drawing library, such as Monogame. In the assignment you have to show that you can use effectively the design patterns learnt during the course. Your GUI library should allow to create at least clickable buttons and text labels organized in a single panel. All assignments are to be presented at the end of the practical assessment to the teacher upon request

At the practicum check you will be asked to solve exercises on design patterns based on the assignment. The maximum score for this part is up to 10 points.
The teachers still reserve the right to check the practicums handed in by each student, and to use it for further evaluation. The university rules on fraude and plagiarism (Hogeschoolgids art. 11.10 -- 11.12) also apply to code.